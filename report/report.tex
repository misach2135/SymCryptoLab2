\documentclass[12pt]{article}

\usepackage{type1ec}
\usepackage{mathtext}
\usepackage[T1, T3]{fontenc}
\usepackage[utf8]{inputenc}
\usepackage[english, ukrainian]{babel}
\usepackage{amsmath}
\usepackage{amssymb}
\usepackage{amsthm}
\usepackage{xcolor}
\usepackage{graphicx}
\usepackage{courier}
\usepackage{listings}
\graphicspath{ {images/} }


\usepackage{geometry}
\geometry{
    a4paper,
    left = 30mm,
    top = 20mm,
    right = 10mm,
    bottom = 20mm
}

\begin{document}
\pagestyle{empty}

\begin{titlepage}
    \begin{center}

        \textbf{НАЦІОНАЛЬНИЙ ТЕХНІЧНИЙ УНІВЕРСИТЕТ УКРАЇНИ}\\
        “КИЇВСЬКИЙ ПОЛІТЕХНІЧНИЙ ІНСТИТУТ ім. Ігоря Сікорського”\\
        Навчально-науковий фізико-технічний інститут\\
        Кафедра математичних методів захисту інформації

        \vspace{6cm}

        \Large \textbf{Симетрична Криптографія}\\
        Комп'ютерний практикум №2\\
        Варіант - 8

    \end{center}

    \vspace{9cm}
    \begin{flushright}
        Виконали:\\
        Студенти групи ФІ-13\\
        Ісаченко Нікіта\\
        Бондаренко Олександр
    \end{flushright}

    \vspace*{3cm}

    \begin{center}
        2024 р.
    \end{center}
\end{titlepage}

\section{Мета роботи}

Засвоєння методів частотного криптоаналізу. Здобуття навичок роботи та аналізу
потокових шифрів гамування адитивного типу на прикладі шифру Віженера.

\section{Хід роботи}

У ході роботи, нам потрібно було реалізувати шифросистему Віженера.
Перша частина роботи полягала у тому, щоб реалізувати додавання та віднімання
символів за алфавітом. Як і КП-1, ми використовували CP-1251 кодування, про переваги
якого вже зазначали у звіті до попереднього комп'ютерного практикуму.

Сам алгоритм шифрування та дешифрування був не складним, оскільки все що ми робили-
додавали або віднімали літеру тексту до літери ключа.

Щоб перевірити нашу реалізацію, ми взяли один шифротекст, та зашифрували його
7 різними ключами. Результати шифрування та дешифрування наведені у додатках.

Код:

\begin{verbatim}
        char VigenereLab::alphabetAdd(char a, char b)
        {
            if (!isSmallLetter(a) && !isSmallLetter(b)) return ERROR_CHAR;

            char a_num = a - char(224);
            char b_num = b - char(224);

            char res = char(224) + ((a_num + b_num) % 32);

            return res;
        }

        char VigenereLab::alphabetSubstract(char a, char b)
        {
            if (!isSmallLetter(b)) return ERROR_CHAR;

            char reversed_b = 2 * char(224) - b + 32;

            return alphabetAdd(a, reversed_b);

        }

        std::string VigenereLab::Cipher(const std::string& text, const std::string& key) 
        {
        std::string result;
        for (size_t i = 0; i < text.size(); i++) {
            char c = alphabetAdd(text[i], key[i % key.size()]);
            result += c;
        }
        return result;
    }

    std::string VigenereLab::Decipher(const std::string& cipher, const std::string& key) 
    {
        std::string result;
        for (size_t i = 0; i < cipher.size(); i++) {
            char c = alphabetSubstract(cipher[i], key[i % key.size()]);
            result += c;
        }
        return result;
    }
    \end{verbatim}

Наступним етапом був взлам Віженера. Для початку, ми визначили довжину блока, шляхом порівняння
індексів літер на відстанні r. Довжина блокого шифротекстку нашого варіанту -- 8. Для взламу ми використали
два методи, код яких наведений далі:

\begin{verbatim}
    std::string VigenereLab::CeasarVigenreCracker(const std::string& text, int blockSize)
    {
        std::map<char, double> langFreqsPr(FREQ_TABLE);
        std::string key;
        auto langGreatest  = GetMaxPairFromMap(FREQ_TABLE);

        for (int i = 0; i < blockSize; i++)
        {
            std::map<char, int> blockFreqs = getFrequency(GetIthBlock(text, blockSize, i));

            auto blockGreatest = GetMaxPairFromMap(blockFreqs);


            key.push_back(alphabetSubstract(blockGreatest.first, langGreatest.first));

        }

        return key;
    }

    std::string VigenereLab::CrackVigenere(std::string& text, int blockSize)
    {
        std::string key;
        for (int i = 0; i < blockSize; i++)
        {
            std::map<char, double> m;
            auto block = GetIthBlock(text, blockSize, i);
            auto blockFreqs = getFrequency(block);

            for (const auto& g : FREQ_TABLE)
            {
                double Mg = 0;
                for (const auto& t : FREQ_TABLE)
                {
                    char tg = alphabetAdd(t.first, g.first);
                    if (blockFreqs.find(tg) == blockFreqs.end()) continue;
                    Mg += t.second * blockFreqs[tg];
                }
                m[g.first] = Mg;
            }

            auto maxElement = GetMaxPairFromMap(m);
            key.push_back(maxElement.first);
        }

        return key;
    }

    
\end{verbatim}

Ключі, які вийшли в результаті:

\begin{verbatim}
    CeasarVigenreCracker: уланобсеребзяныепуля
    CrackVigenere: улановсеребряныепули
\end{verbatim}

Більш точним виявився CrackVigenere метод, оскільки більше спирається на розподіл частот у блоці дані.
Таким чином, отримати відкритий текст:

\begin{verbatim}
    эта система красного карлика никогда не имела названия только зубодробительно длинный номер в 
    каталоге исследовавший ее киберзонд отметил наличие трех газовых 
    гигантов двух астероидных полей кометного облака и занес все эти данные в сектор второй 
    очереди по мнению инка киберзонда система не представляла никакой ценности для пославших его людей на верное будь у него задействованы контуры второго 
    уровнясамостоятельностииазартаонбыпоспорилсамссобойчтовближайшуютысячулетлюдиздесьнепоявятсяипроспорилбылюдипоявилисьвэтойсистеменечерезтысячулетавсеголишьчерезсемьэтобылинетелюдичтопосылализондформальноонивообщенедолжныбылизнатьосуществованииэтойсистемыноутехктоихпосылалбылиденьгимногоденегисредипрочегоиххватилонаточтобыполучитьвозможностьознакомитьсясрезультатамикартографированиязаинтересовавшегоихсекторатаквсистемепоявиласьстанциянаскоропеределаннаяизсписанногогрузовикаитридесяткабуевраннегооповещенияподсвечивающихпространствоврадиусепятисветоднейотнеечерезнесколькомесяцевнастанциюпришелпервыйкорабльэтобылстранныйкорабльсвидуобычныйдесятикилотонниксотникоторыхлетаюткакповнутренниммаршрутамсолнечнойтакинавнешниеколониинеобычнымжеегоделалисеребристыеовалынабортахпонимающийчеловеклегкобымогопознатьвэтиховалахтяжелыеизлучателимайерсапредставлявшиесобойглавныйкалибркрейсероввксфедерациикорабльбылнеодиндругиепохожиенанегоразвдватримесяцазалеталивсистемудатьотдыхкомандеимеханизмампровестимелкийремонткоторыйотчеготонемогливыполнитьсобственныесервыкораблявпрочемремонтневсегдабылмелкимодинизкораблейприползнастанциюсперекореженнымбортомоставляяпозадитаюшийсиневатыйследсочащейсяизразбитыхотсековатмосферыонявновстретилкоготоравногопосиламаможетбойбылнеравныйноэтотктотознаячтопошадынеприходитсяждатьоченьстаралсяпродатьсвоюжизньподорожетригодаспустясистемунавестилещеодинкиберзондоднакохотяегосканирующиесистемыбылинапорядокмощнеечемупредшественниказадействоватьихоннесталвместоэтогоновыйгостьтихозависнадплоскостьюэклиптикизапределамидосягаемостибуевипринялсявпитыватьинформациюшумсолнечноговетратяжелыйрокотгравитационныхволнпланетобрывкиразговоровмеждустанциейиочереднымприбывающимкораблемпоследнееегоинтересовалоособенносильноаещечерезмесяцвсистемепоявилисьновыекораблипятьузкиххищныхтенейтотчеловекчтомогбыопознатьсеребристыеовалынавернякасумелбыузнатьиихпотомучтомалосчемвовселеннойможноспутатьизящныйпрофильэсминцавкстипасиранотроевновьприбывшихушливбокблокируяточкупереходаадвесеребристыеполоскирванулисьпрямокстанциигдекакраззаканчивалподготовкукполетуочереднойкорабльтемнотавокругтьмаитишинаигдетотамждетнечтоцельмишеньврагоднимсловомточтонадоуничтожитьсправадонессятихийзвуктолискриптолишорохямгновенноотскочилвсторонуиокатилподозрительныйучастоквееромогнятихийтрескэтозвуквыстреловазвонкиеиглухиехлопкиэтошарикиплазмывимитационномрежимезвонкиеобстенуиглухиевмишеньтеоретическиимиможнобылобытемнотуподсвечиватьнопоусловиямзачетаяопасаюсьдемаскировкипотомуплазмачернаявидетьвинфракрасномяпоканенаучилсяавотшорохвпередияпрыгалпокомнатесловноплохаямарионеткапосылаяновуюочередьпреждечемзатихнетпредыдущаяисчиталглухиеударыпадающихтелпятьшестьитемнотазначитещектотоосталсясколькожеихгадовсемьиливосемьяполуприселнаклонилсявпередирастопырилрукисловновсплывшаяжабаточьвточькаккитаезаченьвоназанятияхрасслабилсяислушаешьголосвселеннойсейчасонтебеспоетвухогдепрячетсяпоследняяцельнасамомделеяужедавноубедилсячтоникакимиэкстрапараипрочимисверхспособностяминеобладаюноможнопопытатьсякупитьнаэтотфокусоператораикупилочереднойшорохдонессяиззаспиныеслибыядействительноловилушамиголосиззакраямиратутбымнеибылполныйконецзачетанопосколькуязанималсяловлейисключительнореальныхзвуковтоупалвпередуспевприэтомизвернутьсяипрошитьочередьюпространствопередсобойперекатилсяполучивприэтомчувствительныйударвпоясницупослалвторуюочередьпримернотудакудаипервуюинепрекращаяпалитьповелстволвнизнатотслучайеслигадуспелрастянутьсянаполузачетноеиспытаниеоконченовсемишенипораженывкомнатеначалмедленноразгоратьсясветяпопыталсяприподнятьсясполаисразужесхватилсязаушибленныйживотавотнечегопадатьнаоружиеонокакправилотвердоеиребристоенуикактебекомнатамракаехидноосведомилсяоператормрачнокакмояфамилиянопоследиснейлендамнеуженичегонестрашнотакужинестрашнокогдатвойлучшийдругвылетаетсэкзаменаусловноубитыйпузатойзеленойворонойуженичегохуженебываетнунуладнокурсантсвободенполучаяназадодеждуяобнаружилчтопокаяотстреливалкотоввтемнойкомнатенабрикпоступилосообщениеинтереснооткогоэхвотбыотджейнтретийсвободныйуикэндинескемпровестиобидновольнослушателювукомраковичунемедленноявитьсяналейтстриткполковникукоринуоппадааэтонеджейнналейтстритразмещалосьместноеотделениеконторыкоторуювсесодружествокосоухмыляясьименовалоконторойглубинногобуренияхотянаэтомзданиивиселатабличкафирмыпоэкспортукокосовыхореховачутьпоодальпанельрекламыпериодическивыплевывающаянастенусоседнегомонодомаслоганкокосыгрузимбыстрооноивидноколониивсистемебезкокосовыхореховневыживутвымрутскореечемотвзрывнойдекомпрессиировночерездвадцатьоднуминутуяробкоподошелкмерцающейдверицельвашеговизитагрознопроревеламозаиканадпроемомтонвопросапредполагалчтоприлюбомнеудовлетворительномответеменяпревратятвоблачкоразогретогопараиподеломпосколькушлятьсяудверейэтойфирмымогуттольколибоеесотрудникилибозлобныеиномиряненуаеслипопадетсякакойтоэкспортеркокосовбываетнеповезлокурсантмраковичкполковникукоринупроблеяляотдушинадеясьчтоинтелктрониканесочтетдрожьвмоемголосехарактернымдляиномирцевпризнакоммерцающаязавесаисчезлапроходитеголососталсятакимжерезкиминеприятнымнопокрайнеймересталнаполтонатишеяосторожноступилнасверкающийполповернитесьлицомкстенесмотритепередсобойпротянитерукувотверстиеанализсетчаткииднкпроверяютилиявсамомделевукомраковичгражданинфедерациидвадцатьпервогогодаотродуилинежитькакаякакговориламояпокойнаячешскаябабушканикогданеслышавшаяпроиномирянследуйтезакраснымсигналомзакакимещекраснымсигналомпоинтересовалсяяотворачиваясьотстеныиуставилсянакрасныйогонеквисевшийввоздухепрямопередмоимлицомследуйтезакраснымсигналомлюбоеотклонениеотмаршрутасчитаетсянарушениемагашагвсторонупобегпрыжокнаместепровокацияэтоужемойрусскийдедушкавывсехтаквстречаетеилитолькоменянапоследокпоинтересовалсяядвинувшисьзаогонькомвсехпостороннихпытающихсяпройтичерезслужебныйвходсообщилголостакиоставивменявнедоумениитолияговорилсвозомнившимосебеинкомтолиссадюгойохранником
\end{verbatim}

\section{Висновок}

У роботі ми оцінили надлишковість російської мови та ентропії на символ та біграм,
що певне знадобиться нам у подальших дослідженнях. Ось така штука.


\end{document}